\documentclass{article}
\usepackage[T1]{fontenc}
\usepackage[polish]{babel}
\usepackage[utf8]{inputenc}
\usepackage{graphicx} % Required for inserting images

\title{SymPhys} %placeholder
\author{Hubert Krata - gr 2, Wojciech Szot - gr 2, Dawid Zapolski - gr 2}
\date{}

\begin{document}

\maketitle

\section{Opis}
Nasz projekt będzie składał się z zestawu symulacji fizycznych, które umożliwią interaktywne poznawanie różnych zjawisk. Obecnie planowane to:
\begin{itemize}
    \item Ruch punktowych ładunków elektrycznych pod wpływem siły elektrostatycznej + podobnie dla grawitacji
    \item Zderzenia dwóch ciał
    \item Wahadła - matematyczne i sprężynowe, również z oporem ruchu
    \item Ogólna symulacja ruchu jednego ciała pod wpływem działania siły podanej wzorem (zależność od czasu i położenia)
    \item Efekt Dopplera - przejeżdżająca karetka 

    
\end{itemize}

\subsection{Funkcjonalność}

Po uruchomieniu programu, z listy będzie można wybrać jedną z kilku symulacji.
Będą one umożliwiały zmianę parametrów symulacji, takich jak masa, ładunek elektryczny, prędkość czy przyspieszenie. Możliwe będzie też przyspieszanie i spowalnianie symulacji. 

Każda z symulacji będzie przebiegała wobec schematu - wybór parametrów, start symulacji, możliwość zatrzymania w dowolnym momencie, możliwość zmiany parametrów i rozpoczęcia od początku. 
Można będzie również wykonać wykres w zależności od czasu dla położenia, prędkości i przyspieszenia.



\section{Biblioteki}
Mamy zamiar korzystać z JavaFX jako podstawy graficznej, z dodatkiem charts w celu wykonywania wykresów.

\end{document}
